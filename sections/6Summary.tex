\section{Summary}
Studies of the nuclear force are a fascinating yet unfinished part of physics.
Though most simple interactions involving it are well understood, it is yet to be applied to more complicated systems.
Mass spectrometers have been crucial in testing nuclear theories and pushing their limits.
They have allowed for the exploration of the nuclear landscape closer to the neutron dripline where models using only 2N interactions have been insufficient in predicting the shell closures of these exotic nuclei.
Future MS such as the PENTATRAP experiment \cite{repp_pentatrap_2012} or the many MR-TOF spectrometers that are being developed, with improved resolving power and shorter needed observation time will be crucial in probing and improving the leading nuclear force theories.

% In summary, the current research outlines the role of mass spectrometers in ongoing work in the field of nuclear physics.
% Nuclear physicists works on determining different masses of the nucleus and different isotopes and works on the fundamentals; we can build on a better future through this work.
% However, it is clear that there are still much to prove and disproof, and currently, we do not know how much is unknown.
% Nevertheless, there are many applications of mass spectrometers we are happy with; from water treatment to uses in space exploration, the technology of MS had a considerable impact on our lives.