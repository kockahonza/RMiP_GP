\section{Nuclei mass measurements with regard to nuclear force theory}
The binding energy is the energy required to separate the protons and neutrons.
We can determine the binding energy from its rest mass using Einstein’s famous equation $E=mc^2$.
A bounded system has a smaller mass than when it is separated.
Work is done to separate the systems; therefore, energy is put into the system.
When separated, the particles are at rest.
Consequently, if we can measure the rest mass, we would find that the rest mass has increased. [1]
We find that the binding energy value is approximately proportional to the number of nucleons for any nucleus when looking at the binding energy.

Besides the standard investigations, we can also investigate the Neutron pairing energy of finite nuclei.
We can study a range of odd and even pairs of N/Z numbers; to investigate the strong nuclear force.
(N is the number of Neutron and Z is the number of protons) [2] So far, the idea is that the (N)even-(Z)even nucleus (although energy depends on factors such as the kind of particles and state it is occupied) we known for a fact that odd(N)- odd(A) nucleus is 1/2 to 2/3 times smaller, this is when they are given in the same shell and the mass number A is very close to one another.
This strange character is due to the nucleon-nucleon potential resulting from the strong force. [3]

There is a model known as the liquid drop model, which underpredicts the binding energy of magic nuclei, magic nuclei referrer to nuclei of N (neutron number) or Z (proton number) equating to either of the following numbers, 2, 8, 20, 28, 50, 82, 126.
The neutron/ proton separation energy peaks if N(Z) equals a magic number; on the zigzag graph, we see this as the last point before the significant drop.
There is a more stable isotope if Z is a magic number and a more stable isotone if N is a magic number.
If either N or Z or both are magic numbers, then the energies of the excited state will be much higher than the ground state.
Another discovery is that elements with Z equal to a magic number have a more prominent natural abundance than nearby elements. [4]

The magic number can further be explained by the shell closure model of the nucleus.
This is done by considering each nucleon to be moving in some potential and classifying the energy level in terms of quantum numbers n l j, like how the wavefunction of individual electrons are classified in atomic physics.
The energy eigenvalues depend on the principal quantum number, n, orbital angular momentum.
The energy level comes in shells, with a large gap just above each shell. [5]

In particle physics, we often use the idea of drip lines to categorize our particles.
We have a one or two-particle drip line, which is a result of the idea that odd and even nucleon numbers, as we know, it has a significant effect on binding energy.
We will be looking at one particle drip line for an odd(Z) or odd(N) nuclei.
Two particle drip line occurs when the energy of the separation of two-particle becomes negative.
Experimentally, we determined the one and two neutron drip lines up to neon. [6]
