\section{Applications of MS outside nuclear force theory}

Mass spectrometry has a long history and wide adoption out with nuclear physics both in research and commercially.
MS are essential for chemical analysis – often used to determine the isotopes present in a mixture.
In chemical and biomedical research, they are used frequently for structural elucidation where MS can give information about molecular structure. \cite{bhattarai_chapter_2020}

\subsection{Penning traps and neutrino physics}
One exciting area of study where MS are fundamental is neutrino physics.
Penning traps are especially well suited to this study due to their high precision 
The neutrino is the most abundant particle with mass - roughly a thousand trillion pass through your body every second. \cite{noauthor_whats_nodate}
Yet the incredibly light particles are elusive and difficult to detect as they only interact with the weak force and gravity so surprisingly little is known about them. \cite{noauthor_what_nodate}
Research in neutrino science could further the understanding of the standard model and they could be important in explaining why there is matter in the universe and not antimatter \cite{gibney_morphing_2015} – it is thought that neutrinos could be their own antimatter, unlike any other fermion in the standard model [8].
Another interesting behaviour they have is neutrino oscillometry where the particle can change flavour (electron, muon, tau) as it travels.
It was this behaviour which proved that not only do neutrinos have mass but that they are a quantum superposition of 3 masses \cite{gibney_morphing_2015}.
Since neutrinos are chargeless themselves, an upper limit to their rest mass can be measured by the mass difference between mother and daughter nucleotides from a nuclear decay. \cite{eliseev_penning-trap_2013}
For example the $\beta$ -decay: $^{187}Re \longrightarrow ^{187}Os + e^- + \bar{\nu _e}$ with Rhenium as the mother nucleotide and Osmium the daughter. \cite{repp_pentatrap_2012}
The current upper limit of this mass is 1.
1eV measured at the Karlsruhe Tritium Neutrino (KATRIN) spectrometer. \cite{castelvecchi_physicists_2019}
Although KATRIN is an electron spectrometer and not a MS, PENTATRAP is a new five Penning trap facility being built in Heidelberg, Germany and is expected to reduce this upper mass limit even further \cite{repp_pentatrap_2012}.

\subsection{MR-TOF's applications outside physics}
MR-TOF MS also have many applications outside nuclear physics.
Due to their compact size and relative affordability compared to Penning traps, MR-TOF's are suited to \emph{in situ} applications \cite{dickel_multiple-reflection_2013}.
One potential application is waste water management where the MS could identify pollutants \cite{dickel_multiple-reflection_2013} – during COVID-19 pandemic waste water analysis was used to model the spread of the disease \cite{noauthor_wastewater_nodate}.
One current application of the technology is in biomedical analysis where a variation of the device (MALDI-TOF MS) is used routinely as an accurate and fast method to identify of microorganisms which reduces costs as it requires minimum consumables. \cite{noauthor_uk_2019}